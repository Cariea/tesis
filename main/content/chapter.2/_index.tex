\uchapter{Marco Teórico}

Este capítulo presenta las bases teóricas y las investigaciones previas que sustentan el desarrollo del Sistema de Monitoreo Acústico para Identificar Sonidos y Generar Alertas de Emergencia. A través de la revisión de fuentes documentales, se exponen los conceptos y enfoques principales relacionados con el procesamiento de señales acústicas, la clasificación de sonidos mediante técnicas de inteligencia artificial y los sistemas de alertas automatizadas.

El marco teórico no solo proporciona un contexto académico para el proyecto, sino que también justifica las decisiones técnicas y metodológicas adoptadas en el diseño del sistema. En las siguientes secciones se detallan los antecedentes investigativos, los modelos de IA aplicados a la clasificación de sonidos y los fundamentos teóricos que permiten la generación de alertas en situaciones de emergencia.

En las siguientes secciones se detallan los antecedentes y los conceptos clave que sustentan este trabajo.


\usection{Antecedentes de la Investigación}

\usubsection{Real-time Audio Classification on an Edge Device [Clasificación de audio en tiempo real en un dispositivo perimetral]}

\citeauthor{malmberg_real-time_2021} \citeyear{malmberg_real-time_2021} Abordan la implementación de modelos de aprendizaje automatico en dispositivos edge para la clasificación de audio en tiempo real, Se centra en el uso del modelo YAMNet, que fue reentrenado para detectar eventos acústicos como disparos, rotura de cristales, animales y habla humana. Este modelo puede desplegarse en dispositivos edge en su versión completa con TensorFlow como en versiones optimizadas con TensorFlow Lite, con el objetivo de comparar la precisión, el tiempo de inferencia y el uso de memoria de cada variante.

Con el propósito de evaluar el desempeño, trabajaron en una serie de experimentos en los que compararon la precisión del modelo en ambas versiones de TensorFlow. Se encontraron con que, aunque existía una pérdida de precisión en la versión lite los resultados eran comparables a los de la versión completa. Esto implica que TensorFlow Lite es una opción viable y crea la posibilidad de trabajar con dispositivos de bajo consumo y recursos limitados como ESP32

Este enfoque no solo demuestra la factibilidad de implementar modelos de clasificación de audio en tiempo real en entornos con recursos restringidos, sino que también abre la puerta a futuras investigaciones que busquen optimizar aún más estos modelos. La capacidad de desplegar soluciones de inteligencia artificial en dispositivos edge amplía las posibilidades de aplicaciones en áreas como la seguridad, la vigilancia y el monitoreo ambiental, permitiendo respuestas más rápidas y reduciendo la dependencia de infraestructuras centralizadas.

\usubsection{Analysing RMS and peak values of vibration signals for condition monitoring of wind turbine gearboxes [Análisis de valores RMS y pico de señales de vibración para la monitorización del estado de las cajas de engranajes de aerogeneradores]}

\citeauthor{igba_analysing_2016} \citeyear{igba_analysing_2016} abordan el monitoreo de condición en aerogeneradores con el objetivo de detectar fallos en las cajas de engranajes mediante el análisis de valores RMS y picos de vibraciones. La investigación propone tres modelos: correlación de señales, vibración extrema e intensidad RMS, validados con datos en dominio del tiempo. A través del uso de la teoría de valores extremos, se identificaron indicadores que permiten detectar fallos en etapas tempranas, lo que facilita la planificación del mantenimiento y minimiza el tiempo de inactividad de los aerogeneradores. Los resultados demostraron que el monitoreo de vibraciones proporciona información clave sobre el estado de los componentes y que la precisión de cada técnica depende de la física de la falla, sugiriendo un enfoque integral que combine distintas estrategias para una evaluación más robusta de la salud de los aerogeneradores.

El uso del valor RMS en este estudio es fundamental para la monitorización de la condición de las cajas de engranajes en aerogeneradores, ya que permite evaluar el nivel global de vibración y detectar fallos progresivos, como el desgaste de rodamientos y grietas en ejes. \citeauthor{igba_analysing_2016} \citeyear{igba_analysing_2016} destacan que el RMS es una métrica confiable para identificar tendencias anómalas en las vibraciones, lo que facilita la detección temprana de fallos antes de que se conviertan en problemas críticos. A pesar de ciertas limitaciones, como su menor sensibilidad a fallos incipientes en los dientes de los engranajes, el análisis de RMS sigue siendo un pilar clave en la estrategia de mantenimiento basado en condición (CBM), al proporcionar información valiosa sobre la evolución del estado de los componentes mecánicos.

\usubsection{Metodología para la identificación de eventos sonoros anómalos}

\citeauthor{torija_metodologia_2018} \citeyear{torija_metodologia_2018} presentan una metodología para la detección de eventos sonoros anómalos en entornos urbanos. Su trabajo se centra en el análisis de sucesos acústicos que generan incrementos bruscos de energía sonora en el paisaje sonoro urbano, lo que puede provocar molestias significativas en la población expuesta. Según los autores, los eventos sonoros anómalos, como bocinas, gritos, explosiones o disparos, generan una percepción intensificada del ruido ambiental debido a su impacto en la focalización de la atención. Para abordar este problema, desarrollaron un modelo basado en la evaluación del tiempo de estabilización de la medición del ruido ambiental y el análisis del factor cresta, que permite estimar el incremento de energía sonora generado por estos eventos.

El estudio incluyó mediciones en 35 localizaciones de la ciudad de Granada, utilizando un sonómetro Brüel Kjaer tipo 1. La metodología aplicada permitió definir un evento sonoro anómalo como aquel que provoca un incremento del nivel de energía sonora de al menos un 25\% respecto al nivel de fondo caracterizado por el descriptor LA90.

Los resultados indicaron que el número de eventos sonoros anómalos presentes en una ubicación está altamente correlacionado con la variabilidad de la energía sonora en la zona. Además, se evidenció que el factor cresta es un parámetro clave para estimar la magnitud del impacto acústico.

\usubsection{Edge AI for Real-Time Anomaly Detection in Smart Homes [Inteligencia artificial de borde para la detección de anomalías en tiempo real en hogares inteligentes]}

\citeauthor{reis2025edge} \citeyear{reis2025edge} proponen un marco de trabajo para la detección de anomalías en tiempo real en dispositivos de borde (Edge AI), con el objetivo de superar las limitaciones de latencia y privacidad de los sistemas centralizados en la nube. Su enfoque se basa en una arquitectura híbrida que combina dos modelos de aprendizaje automático, Isolation Forest (IF), un modelo ligero para la detección rápida de eventos puntuales, y un Autoencoder de Memoria a Corto y Largo Plazo (LSTM-AE), un modelo de aprendizaje profundo para identificar patrones anómalos en secuencias de datos.

El sistema fue evaluado en plataformas de borde como Raspberry Pi y NVIDIA Jetson Nano para validar su desempeño en entornos con recursos limitados. Con el propósito de evaluar el rendimiento, los autores compararon la precisión y la eficiencia de ambos modelos. Los resultados demostraron que el modelo LSTM alcanzó una mayor precisión de detección (hasta un 93.6\%), pero con un costo computacional y energético más elevado. Un hallazgo clave fue la aplicación de técnicas de cuantización sobre el modelo LSTM, lo que permitió una reducción del 76\% en el tiempo de inferencia y del 35\% en el consumo de energía. Esto confirma que es viable optimizar modelos complejos para su despliegue eficiente en dispositivos de borde sin una pérdida significativa de rendimiento.

Este enfoque no solo demuestra la factibilidad de implementar sistemas de monitoreo de anomalías en tiempo real que sean rápidos, privados y eficientes energéticamente, sino que también subraya la ventaja de una arquitectura híbrida que equilibra velocidad y precisión. La capacidad de desplegar estas soluciones directamente en el hardware del hogar amplía sus aplicaciones a otros dominios como edificios inteligentes e IoT industrial, reduciendo la dependencia de la infraestructura en la nube y permitiendo respuestas más autónomas ante eventos críticos.


\usection{Bases Teóricas}

\usubsection{Dispositivos Edge}

El incremento en la demanda de los servicios y aplicaciones en las últimas décadas del uso de la Internet, ha contribuido a un fuerte aumento de los requisitos de procesamiento y almacenamiento de datos.  Son diversos, en términos de los recursos que requieren las diferentes aplicaciones y, por lo tanto, a menudo invocan soluciones a medida (Doluí y Kanti Datta, 2017).

Según Medina (2019) Edge Device, en este contexto, se refiere a elementos con capacidades limitadas que tiene su propio conjunto de recursos: CPU, memoria, almacenamiento, y red. Pueden ser Smartphone, Smartglasses, smartwatches, tablets, routers, vehículos autónomos, o cualquier dispositivo de IoT con capacidad de proceso. En este escenario, Edge Computing, representa una solución para enfrentar y aliviar la carga de procesamiento y almacenamiento en la nube, en aplicaciones y tecnologías que requieren de un ancho de banda exponencial y una baja o nula latencia.

Shi, Cao, Zhang, Li y Xu (2016) definen Edge Computing como un paradigma de computación distribuida que acerca el procesamiento y almacenamiento de datos a la fuente de generación, es decir, al ``borde'' de la red. Este enfoque permite reducir la latencia, optimizar el uso del ancho de banda y mejorar la eficiencia en aplicaciones que requieren respuestas en tiempo real, como el Internet de las Cosas (IoT), la realidad aumentada, los vehículos autónomos y las ciudades inteligentes. Edge Computing se basa en dispositivos periféricos (edge devices) y nodos locales que realizan tareas de procesamiento y almacenamiento, lo que reduce la dependencia de la infraestructura centralizada de la nube.


\usubsection{Raspberry pi}

Según Richardson y Wallace (2016), la Raspberry Pi es un dispositivo de computación de bajo costo y alto rendimiento que ha ganado popularidad en diversos campos debido a su versatilidad y facilidad de uso. Este dispositivo, del tamaño de una tarjeta de crédito, está equipado con un procesador ARM, memoria RAM, puertos de entrada/salida y conectividad de red, lo que lo hace ideal para proyectos educativos, de automatización y desarrollo de prototipos. La Raspberry Pi es capaz de ejecutar sistemas operativos basados en Linux, lo que permite a los usuarios programar y personalizar sus aplicaciones según sus necesidades (Richardson y Wallace, 2016).

Además, Upton y Halfacree (2020) destacan que la Raspberry Pi ha evolucionado significativamente desde su lanzamiento, con modelos más potentes como la Raspberry Pi 4, que ofrece mayores capacidades de procesamiento, almacenamiento y conectividad. Este modelo incluye soporte para redes Gigabit Ethernet, puertos USB 3.0 y salidas de video de alta definición, lo que lo convierte en una herramienta poderosa para aplicaciones más exigentes, como servidores domésticos, centros multimedia y sistemas embebidos avanzados (Upton y Halfacree, 2020).

\usubsection{Python}

Python es un lenguaje de programación poderoso, elegante y fácil de leer, diseñado para simplificar la creación de programas mediante una sintaxis clara y estructurada. Según el documento, Python destaca por su versatilidad en aplicaciones del mundo real, su enfoque en la legibilidad del código y su capacidad para integrar paradigmas como la programación orientada a objetos y funcional. Además, es software libre con una comunidad activa y una implementación estándar consolidada (Yuill Halpin, 2006).

\usubsection{Tensorflow}

Según Goldsborought (2016) TensorFlow es una biblioteca de software de deep learning de código abierto desarrollada por Google que permite definir, entrenar y desplegar modelos de machine learning mediante la representación de algoritmos como grafos computacionales.

La librería TensorFlow opera construyendo un grafo computacional en el que cada nodo representa una operación (por ejemplo, una función matemática, una transformación o una capa de una red neuronal) y cada arista transporta un tensor, es decir, un arreglo multidimensional de datos. Esta arquitectura facilita diversas optimizaciones, como la eliminación de subgrafos redundantes, y permite distribuir la ejecución de la computación a lo largo de múltiples dispositivos (CPUs, GPUs, TPUs) e incluso en entornos distribuidos. De esta manera, se optimiza tanto el uso de memoria como el rendimiento, haciendo viable el entrenamiento y despliegue de modelos complejos a gran escala. (Goldsborought, 2016).

\usubsection{NestJs}

Según Sabo (2020) NestJS es un framework para el desarrollo de aplicaciones del lado del servidor basado en Node.js, que se escribe en TypeScript. Proporciona una estructura modular y escalable mediante el uso de patrones modernos como la inyección de dependencias, controladores y módulos, facilitando la creación de aplicaciones backend mantenibles y robustas.

NestJS aprovecha la solidez de Node.js y Express.js, pero se diferencia al introducir un enfoque inspirado en Angular para la organización de la aplicación. Gracias a su arquitectura basada en módulos, cada parte de la aplicación se encapsula en unidades independientes que facilitan la reutilización y la escalabilidad. Además, el framework implementa un avanzado sistema de inyección de dependencias, lo que permite gestionar y suministrar las instancias de servicios de manera automática, reduciendo el acoplamiento entre componentes y promoviendo un diseño orientado a pruebas. Otro pilar fundamental de NestJS es su fuerte integración con TypeScript, lo cual aporta tipificación estática y facilita la detección temprana de errores durante el desarrollo. La utilización de decoradores en NestJS permite agregar metadatos a clases, métodos y propiedades, lo que habilita la implementación de características como interceptores, pipes y controladores para la validación y transformación de datos, así como para el manejo de rutas HTTP. Además, la herramienta Nest CLI agiliza la generación de nuevos proyectos y componentes, garantizando que se siga una estructura coherente en toda la aplicación, lo que resulta especialmente útil en proyectos complejos y colaborativos (Sabo, 2020).

\usubsection{Inteligencia artificial}

La Inteligencia Artificial (IA) se define como el estudio de agentes que perciben su entorno a través de sensores y actúan sobre él mediante actuadores, con el objetivo de maximizar su utilidad esperada. Según Russell y Norvig (2022), `` La IA es el estudio de agentes que reciben percepciones del entorno y realizan acciones. Cada uno de estos agentes implementa una función que asigna secuencias de percepción a acciones, y cubrimos diferentes formas de representar estas funciones para lograr el mejor resultado esperado.'' (p. 19).

Los fundamentos de la IA se entrelazan con múltiples disciplinas. La filosofía aporta marcos éticos y lógicos, como señalan los autores: ``El filósofo griego Aristóteles fue uno de los primeros en intentar codificar el ``pensamiento correcto'' sus silogismos proporcionaron patrones para estructuras argumentales que siempre produjeron conclusiones correctas.'' (p. 21). Las matemáticas y la estadística proporcionan herramientas para el razonamiento probabilístico: ``La probabilidad rápidamente se convirtió en una parte invaluable de las ciencias cuantitativas, ayudando a lidiar con mediciones inciertas y teorías incompletas.'' (p. 26). La economía contribuye con teorías de decisión y utilidad: ``La teoría de la decisión, que combina la teoría de la probabilidad con la teoría de la utilidad, proporciona un marco formal y completo para las decisiones individuales tomadas en condiciones de incertidumbre.'' (p. 28). La neurociencia y la psicología inspiran modelos cognitivos: ``El campo interdisciplinario de la ciencia cognitiva reúne modelos informáticos de la IA y técnicas experimentales de la psicología para construir teorías precisas y comprobables de la mente humana.'' (p. 21). Por último, la ingeniería y la computación permiten implementar sistemas eficientes: ``La historia de la IA es también la historia del diseño de arquitecturas cada vez más sofisticadas para programas de agentes.'' (p. 65).

La IA actual ha alcanzado hitos significativos en diversos dominios. De acuerdo a los expuesto por Russell Norvig (2022): “Los sistemas que usan IA han alcanzado o superado el rendimiento humano en ajedrez, Go, póquer, Pac-Man, Jeopardy, detección de objetos ImageNet, reconocimiento de voz y diagnóstico de retinopatía diabética.'' (p. 46). En aplicaciones prácticas, destacan los vehículos autónomos: ``Los vehículos de prueba de Waymo superaron la marca de 10 millones de millas recorridas en vías públicas sin sufrir accidentes graves.'' (p. 47), y sistemas de diagnóstico médico: ``Los algoritmos de IA ahora igualan o superan a los médicos expertos en el diagnóstico de muchas afecciones, como el cáncer metastásico y las enfermedades oftálmicas.'' (p. 48). Además, herramientas como ``Los sistemas de traducción automática ahora permiten la lectura de documentos en más de 100 idiomas, lo que genera cientos de miles de millones de palabras por día.'' (p. 47) evidencian su impacto global. No obstante, persisten retos éticos y técnicos, como la alineación de valores humanos y la escalabilidad en entornos complejos, que definen la frontera actual de investigación.

\usubsection{Cadenas de Markov}

Fundamentos teóricos Las Cadenas de Markov, nombradas en honor al matemático ruso Andrei Andreevich Markov (1856–1922), son procesos estocásticos que modelan sistemas donde el futuro depende únicamente del estado presente, sin influencia directa del pasado (Matas Soberón, s.f., p. 13). Este principio, conocido como propiedad de Markov, fue inicialmente aplicado por Markov en el análisis de secuencias de vocales y consonantes en textos literarios, como Eugene Onegin de Pushkin, sentando las bases para su uso en campos como la física, biología y ciencias sociales (Matas Soberón, s.f., p. 11). Posteriormente, estas cadenas han sido utilizadas para modelar sistemas en epidemiología, teoría de colas y dinámica de poblaciones (Bobadilla Osses, 2010, p. 5).

Una Cadena de Markov se define formalmente como un proceso estocástico ${X_n}$ sobre un espacio de estados S (finito o numerable), donde la probabilidad de transición al siguiente estado depende exclusivamente del estado actual. Esta propiedad se traduce en una matriz de transición $P = (p_{ij})$, donde cada entrada $p_{ij}$ representa la probabilidad de pasar del estado i al estado j. La matriz P es estocástica, es decir, sus filas suman 1 y sus elementos son no negativos (Matas Soberón, s.f., p. 14-15). Además, en sistemas más avanzados, se pueden considerar Cadenas de Markov de tiempo continuo, las cuales presentan una relación con la teoría de semigrupos de operadores (Bobadilla Osses, 2010, p. 6).

Las cadenas de Markov pueden clasificarse en homogéneas, cuando la matriz de transición P permanece constante en el tiempo (Matas Soberón, s.f., p. 23). Un estado es recurrente si la cadena regresa a él infinitas veces, y transitorio si eventualmente lo abandona para siempre. Esto se determina mediante la representación canónica de P, que identifica clases cerradas (conjuntos de estados que no pueden abandonarse) (Matas Soberón, s.f., p. 24-25). En el contexto de sistemas biológicos, por ejemplo, una cadena de Markov puede utilizarse para modelar la propagación de una enfermedad dentro de una población, donde los estados pueden representar niveles de infección y la recurrencia indicar posibles rebrotes (Bobadilla Osses, 2010, p. 91).

Para cadenas regulares (matrices P con todas sus entradas positivas en alguna potencia), las probabilidades convergen a un estado estacionario w, único vector estocástico que satisface w = wP. Este resultado se fundamenta en el teorema de Perron-Frobenius y técnicas como la descomposición de Jordan (Matas Soberón, s.f., p. 29-30). Un caso particular de convergencia ocurre en modelos de colas, donde la distribución estacionaria describe la cantidad promedio de clientes en espera en el largo plazo (Bobadilla Osses, 2010, p. 84).

Los autores en sus respectivas publicaciones ilustran las utilidades de las Cadenas de Markov en casos reales. En fútbol, se utilizan para modelizar resultados (ganar, empatar, perder) en partidos de la liga española, calculando probabilidades de equilibrio para equipos ganadores y perdedores (Matas Soberón, s.f., p. 33-41). En póquer, se emplean para el análisis de rondas de apuestas y distribución de cartas, identificando estados recurrentes (como abandonar una partida) mediante matrices de transición (Matas Soberón, s.f., p. 42-48). En Google AdWords, se aplican para predecir el comportamiento de clics en anuncios, utilizando cadenas regulares para optimizar estrategias de marketing (Matas Soberón, s.f., p. 49-53). También se usan en el modelo de Reed-Frost en epidemiología para estudiar la propagación de enfermedades infecciosas, clasificando estados en susceptibles, infectados y recuperados (Bobadilla Osses, 2010, p. 91-107). Otro ejemplo es su aplicación en cadenas de colas, modelos de atención en sistemas de servicio como supermercados o sistemas de telecomunicaciones, donde la matriz de transición describe la dinámica de llegada y atención de clientes (Bobadilla Osses, 2010, p. 57-61).

\usubsection{Agente}

Según Russell y Norvig (2022), un agente es cualquier entidad que percibe su entorno a través de sensores y actúa sobre él mediante actuadores. En este sentido, un agente puede ser un ser humano, un robot o un sistema de software, siempre que cumpla con esta función de percepción y acción. Por ejemplo, un agente humano posee sensores como los ojos y oídos, y actuadores como las manos y las piernas. Un agente robótico puede incluir cámaras y sensores infrarrojos como entrada, y motores como salida. Un software, en cambio, percibe datos en forma de archivos o paquetes de red y responde modificando archivos o enviando información a otros sistemas (Russell Norvig, 2022).

El comportamiento de un agente está determinado por su función de agente, la cual define cómo actúa en función de la secuencia de percepciones que ha experimentado. En otras palabras, un agente toma decisiones basándose en su conocimiento previo y en los datos que recibe en tiempo real. Russell y Norvig (2022) también introducen el concepto de agente racional, que es aquel que elige acciones que maximizan su desempeño según un criterio predefinido. La complejidad del entorno en el que opera el agente influye directamente en su diseño y en su capacidad de tomar decisiones óptimas (Russell Norvig, 2022).

\usubsection{Ciencia de datos}

Según Provost y Fawcett (2013), la ciencia de datos se enfoca en el uso de datos para tomar decisiones informadas, utilizando herramientas como el aprendizaje automático, la minería de datos y el análisis estadístico. Los autores destacan que la ciencia de datos no solo se trata de manejar grandes volúmenes de datos, sino también de entender cómo estos pueden ser utilizados para generar valor en diferentes contextos, como en el ámbito empresarial, científico o social. Por su parte, Russell y Norvig (2022) enfatizan que la ciencia de datos es una disciplina que se apoya en la inteligencia artificial y el aprendizaje automático para modelar y predecir comportamientos a partir de datos. Los autores mencionan que la ciencia de datos es fundamental en la creación de sistemas inteligentes que pueden aprender de los datos y mejorar su desempeño con el tiempo. Además, resaltan la importancia de la ética en el manejo de datos, especialmente en aplicaciones que involucran la privacidad y la seguridad de las personas. García, Molina y Berlanga (2018) complementan esta definición al señalar que la ciencia de datos es una disciplina que integra técnicas analíticas avanzadas, como el aprendizaje estadístico y la minería de datos, para resolver problemas complejos. Los autores destacan que la ciencia de datos no solo se limita al análisis de datos, sino que también incluye la preparación, limpieza y transformación de los datos para que puedan ser utilizados en modelos predictivos y descriptivos. Además, resaltan la importancia de la visualización de datos como una herramienta clave para comunicar los resultados de manera efectiva.

En el contexto del Sistema de Monitoreo Acústico para Identificar Sonidos y Generar Alertas de Emergencia, la ciencia de datos juega un papel fundamental en varias etapas del proyecto. En primer lugar, se utiliza para el procesamiento y análisis de señales acústicas, donde se aplican técnicas de aprendizaje automático para clasificar sonidos ambientales y detectar eventos anómalos. Como mencionan Provost y Fawcett (2013), el aprendizaje automático es una herramienta poderosa para la clasificación de datos, especialmente en aplicaciones que requieren respuestas en tiempo real, como es el caso de este sistema.

Además, la ciencia de datos es esencial en la creación y entrenamiento de modelos predictivos que permiten identificar patrones de sonidos asociados a situaciones de emergencia. Russell y Norvig (2022) destacan que los modelos de inteligencia artificial, como las redes neuronales y las cadenas de Markov, son fundamentales para la predicción y clasificación de eventos basados en datos históricos. En este proyecto, se utilizan modelos preentrenados, como YAMNet, y se adaptan para la detección de sonidos específicos, lo que demuestra la aplicación práctica de la ciencia de datos en la creación de sistemas inteligentes. Por último, García, Molina y Berlanga (2018) resaltan la importancia de la visualización de datos en la comunicación de resultados. En este proyecto, la ciencia de datos no solo se limita al análisis de sonidos, sino que también incluye la generación de alertas visuales y notificaciones que permiten a los usuarios entender rápidamente la situación y tomar acciones adecuadas. Esto demuestra cómo la ciencia de datos puede ser utilizada para mejorar la usabilidad y efectividad de los sistemas de monitoreo.

\usubsection{Técnicas predictivas para series temporales}

Según Hyndman y Athanasopoulos (2018), las series temporales son secuencias de datos ordenados en el tiempo, y su análisis implica la identificación de patrones como tendencias, estacionalidad y ciclos, para luego utilizarlos en la predicción de valores futuros. Las técnicas predictivas para series temporales son métodos estadísticos y de aprendizaje automático que permiten modelar y predecir el comportamiento futuro de datos que varían en el tiempo. Estas técnicas son fundamentales en aplicaciones donde es necesario anticipar tendencias, patrones o eventos futuros basados en datos históricos.

transformers, lstm, autoencoder, isolation forest, encoding, redis pub/sub (colas),


\usection{Bases Legales}

El desarrollo del Sistema de Monitoreo Acústico se fundamenta en el cumplimiento de las normativas venezolanas vigentes relacionadas con la privacidad, protección de datos y seguridad informática. A continuación, se detallan los instrumentos legales aplicables:

\begin{enumerate}
  \item Constitución de la República Bolivariana de Venezuela (1999)
        \begin{itemize}
          \item Artículo 60: Garantiza el derecho a la protección de la vida privada, intimidad, honor, propia imagen, confidencialidad y reputación. El sistema respeta este principio al no almacenar grabaciones de audio ni recopilar datos personales sensibles, limitándose al análisis acústico en tiempo real y a la generación de alertas basadas en patrones predefinidos.
          \item Artículo 28: Establece el derecho a la protección de datos personales. El sistema utiliza una base de datos local, sin transmisión a la nube, asegurando que la información procesada (como eventos acústicos clasificados) no permita identificar directamente a los usuarios.
        \end{itemize}
  \item Ley Especial contra Delitos Informáticos (2001)
        \begin{itemize}
          \item Artículo 6: Prohíbe el acceso no autorizado a sistemas informáticos. El diseño del sistema incluye medidas de autenticación y encriptación para proteger el acceso al servidor local y evitar intrusiones externas.
          \item Artículo 20: Sanciona la violación de la privacidad mediante la interceptación de comunicaciones. Al no grabar ni almacenar audios, el sistema evita cualquier forma de interceptación ilegítima de información personal.
        \end{itemize}
  \item Ley de Mensajes de Datos y Firmas Electrónicas (2001)
        \begin{itemize}
          \item Artículo 10: Establece la validez jurídica de los mensajes de datos. Aunque el sistema no genera documentos electrónicos, garantiza la integridad de los registros de eventos mediante algoritmos de hash, asegurando la autenticidad de las alertas generadas.
        \end{itemize}
  \item Normas COVENIN
        \begin{itemize}
          \item Norma COVENIN 2500-1: Especifica requisitos técnicos para sistemas electrónicos de seguridad. El sistema cumple con estándares de fiabilidad y funcionalidad, garantizando un monitoreo continuo sin comprometer la privacidad.
          \item Norma COVENIN 27001: Orienta sobre gestión de seguridad de la información. El servidor local implementa protocolos de seguridad física y lógica, como firewalls y restricciones de acceso, para proteger los datos procesados.
        \end{itemize}
  \item Ley Orgánica de Telecomunicaciones (2010)
        \begin{itemize}
          \item Artículo 3: Promueve el uso ético de las telecomunicaciones. Al operar en una red local sin dependencia de servicios externos, el sistema evita riesgos asociados a la transmisión de datos sensibles a través de redes públicas.
        \end{itemize}
\end{enumerate}

El proyecto prioriza la privacidad mediante:
\begin{itemize}
  \item No almacenamiento de audios: Los sonidos capturados se procesan en tiempo real y se descartan inmediatamente después de su análisis.
  \item Base de datos local: La información técnica (como patrones de sonido y registros de alertas) se almacena en un servidor interno, sin conexión a internet, cumpliendo con el principio de confidencialidad.
  \item Transparencia: Los usuarios son informados sobre el funcionamiento del sistema y su finalidad, asegurando consentimiento informado conforme al artículo 60 constitucional.
\end{itemize}

