\usection{Bases Legales}

El desarrollo del Sistema de Monitoreo Acústico se fundamenta en el cumplimiento de las normativas venezolanas vigentes relacionadas con la privacidad, protección de datos y seguridad informática. A continuación, se detallan los instrumentos legales aplicables:

\begin{enumerate}
  \item Constitución de la República Bolivariana de Venezuela (1999)
        \begin{itemize}
          \item Artículo 60: Garantiza el derecho a la protección de la vida privada, intimidad, honor, propia imagen, confidencialidad y reputación.  
          \item Artículo 28: Establece el derecho a la protección de datos personales.
        \end{itemize}
  \item Ley Especial contra Delitos Informáticos (2001)
        \begin{itemize}
          \item Artículo 6: Prohíbe el acceso no autorizado a sistemas informáticos.
          \item Artículo 20: Sanciona la violación de la privacidad mediante la interceptación de comunicaciones.
        \end{itemize}
  \item Ley Orgánica de Telecomunicaciones (2010)
        \begin{itemize}
          \item Artículo 3: Promueve el uso ético de las telecomunicaciones.
        \end{itemize}
\end{enumerate}

% El proyecto prioriza la privacidad mediante:
% \begin{itemize}
%   \item No almacenamiento de audios: Los sonidos capturados se procesan en tiempo real y se descartan inmediatamente después de su análisis.
%   \item Base de datos local: La información técnica (como patrones de sonido y registros de alertas) se almacena en un servidor interno, sin conexión a internet, cumpliendo con el principio de confidencialidad.
%   \item Transparencia: Los usuarios son informados sobre el funcionamiento del sistema y su finalidad, asegurando consentimiento informado conforme al artículo 60 constitucional.
% \end{itemize}
