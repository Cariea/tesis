\usection{Antecedentes de la Investigación}

% TODO - Escribir en españoL el titulo de los antecedentes
\usubsection{Real-time Audio Classification on an Edge Device}

Christoffer y David (2021) Abordan la implementación de modelos de aprendizaje automatico en dispositivos edge para la clasificación de audio en tiempo real, Se centra en el uso del modelo YAMNet, que fue reentrenado para detectar eventos acústicos como disparos, rotura de cristales, animales y habla humana. Este modelo puede desplegarse en dispositivos edge en su versión completa con TensorFlow como en versiones optimizadas con TensorFlow Lite, con el objetivo de comparar la precisión, el tiempo de inferencia y el uso de memoria de cada variante.

Con el propósito de evaluar el desempeño, trabajaron en una serie de experimentos en los que compararon la precisión del modelo en ambas versiones de TensorFlow. Se encontraron con que, aunque existía una pérdida de precisión en la versión lite los resultados eran comparables a los de la versión completa. Esto implica que TensorFlow Lite es una opción viable y crea la posibilidad de trabajar con dispositivos de bajo consumo y recursos limitados como ESP32

Este enfoque no solo demuestra la factibilidad de implementar modelos de clasificación de audio en tiempo real en entornos con recursos restringidos, sino que también abre la puerta a futuras investigaciones que busquen optimizar aún más estos modelos. La capacidad de desplegar soluciones de inteligencia artificial en dispositivos edge amplía las posibilidades de aplicaciones en áreas como la seguridad, la vigilancia y el monitoreo ambiental, permitiendo respuestas más rápidas y reduciendo la dependencia de infraestructuras centralizadas.

\usubsection{Analysing RMS and peak values of vibration signals for condition monitoring of wind turbine gearboxes}

Igba, Alemzadeh, Durugbo Eiriksson (2016) abordan el monitoreo de condición en aerogeneradores con el objetivo de detectar fallos en las cajas de engranajes mediante el análisis de valores RMS y picos de vibraciones. La investigación propone tres modelos: correlación de señales, vibración extrema e intensidad RMS, validados con datos en dominio del tiempo. A través del uso de la teoría de valores extremos, se identificaron indicadores que permiten detectar fallos en etapas tempranas, lo que facilita la planificación del mantenimiento y minimiza el tiempo de inactividad de los aerogeneradores. Los resultados demostraron que el monitoreo de vibraciones proporciona información clave sobre el estado de los componentes y que la precisión de cada técnica depende de la física de la falla, sugiriendo un enfoque integral que combine distintas estrategias para una evaluación más robusta de la salud de los aerogeneradores.

El uso del valor RMS en este estudio es fundamental para la monitorización de la condición de las cajas de engranajes en aerogeneradores, ya que permite evaluar el nivel global de vibración y detectar fallos progresivos, como el desgaste de rodamientos y grietas en ejes. Igba, Alemzadeh, Durugbo Eiriksson (2016) destacan que el RMS es una métrica confiable para identificar tendencias anómalas en las vibraciones, lo que facilita la detección temprana de fallos antes de que se conviertan en problemas críticos. A pesar de ciertas limitaciones, como su menor sensibilidad a fallos incipientes en los dientes de los engranajes, el análisis de RMS sigue siendo un pilar clave en la estrategia de mantenimiento basado en condición (CBM), al proporcionar información valiosa sobre la evolución del estado de los componentes mecánicos.

\usubsection{Metodología para la identificación de eventos sonoros anómalos}

Torija, Ruiz y Ramos-Ridao (2008) presentan una metodología para la detección de eventos sonoros anómalos en entornos urbanos. Su trabajo se centra en el análisis de sucesos acústicos que generan incrementos bruscos de energía sonora en el paisaje sonoro urbano, lo que puede provocar molestias significativas en la población expuesta. Según los autores, los eventos sonoros anómalos, como bocinas, gritos, explosiones o disparos, generan una percepción intensificada del ruido ambiental debido a su impacto en la focalización de la atención. Para abordar este problema, desarrollaron un modelo basado en la evaluación del tiempo de estabilización de la medición del ruido ambiental y el análisis del factor cresta, que permite estimar el incremento de energía sonora generado por estos eventos.

El estudio incluyó mediciones en 35 localizaciones de la ciudad de Granada, utilizando un sonómetro Brüel Kjaer tipo 1. La metodología aplicada permitió definir un evento sonoro anómalo como aquel que provoca un incremento del nivel de energía sonora de al menos un 25\% respecto al nivel de fondo caracterizado por el descriptor LA90.

Los resultados indicaron que el número de eventos sonoros anómalos presentes en una ubicación está altamente correlacionado con la variabilidad de la energía sonora en la zona. Además, se evidenció que el factor cresta es un parámetro clave para estimar la magnitud del impacto acústico.

\usubsection{Edge AI for Real-Time Anomaly Detection in Smart Homes}

Reis y Serôdio (2025) proponen un marco de trabajo para la detección de anomalías en tiempo real en dispositivos de borde (Edge AI), con el objetivo de superar las limitaciones de latencia y privacidad de los sistemas centralizados en la nube. Su enfoque se basa en una arquitectura híbrida que combina dos modelos de aprendizaje automático, Isolation Forest (IF), un modelo ligero para la detección rápida de eventos puntuales, y un Autoencoder de Memoria a Corto y Largo Plazo (LSTM-AE), un modelo de aprendizaje profundo para identificar patrones anómalos en secuencias de datos.

El sistema fue evaluado en plataformas de borde como Raspberry Pi y NVIDIA Jetson Nano para validar su desempeño en entornos con recursos limitados. Con el propósito de evaluar el rendimiento, los autores compararon la precisión y la eficiencia de ambos modelos. Los resultados demostraron que el modelo LSTM alcanzó una mayor precisión de detección (hasta un 93.6\%), pero con un costo computacional y energético más elevado. Un hallazgo clave fue la aplicación de técnicas de cuantización sobre el modelo LSTM, lo que permitió una reducción del 76\% en el tiempo de inferencia y del 35\% en el consumo de energía. Esto confirma que es viable optimizar modelos complejos para su despliegue eficiente en dispositivos de borde sin una pérdida significativa de rendimiento.

Este enfoque no solo demuestra la factibilidad de implementar sistemas de monitoreo de anomalías en tiempo real que sean rápidos, privados y eficientes energéticamente, sino que también subraya la ventaja de una arquitectura híbrida que equilibra velocidad y precisión. La capacidad de desplegar estas soluciones directamente en el hardware del hogar amplía sus aplicaciones a otros dominios como edificios inteligentes e IoT industrial, reduciendo la dependencia de la infraestructura en la nube y permitiendo respuestas más autónomas ante eventos críticos.
