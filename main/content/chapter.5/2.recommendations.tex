\usection{Recomendaciones}

\begin{itemize}
  \item Incorporar modelos de aprendizaje secuencial más avanzados: La integración de arquitecturas como transformers adaptados al procesamiento de audio permitiría mejorar la comprensión del contexto acústico y reducir la incidencia de falsos positivos o negativos en escenarios ambiguos.
  \item Desarrollar interfaces más intuitivas y accesibles: Se sugiere implementar una interfaz de usuario que permita a los usuarios configurar umbrales de alerta, definir eventos críticos personalizados y gestionar contactos de emergencia de forma amigable, incluyendo opciones accesibles para personas mayores.
  \item Implementar una funcionalidad de alarma inteligente basada en contexto y perfil de actividad: se recomienda el desarrollo de una funcionalidad de alarma inteligente que no solo se active ante eventos sonoros abruptos, sino que también considere el contexto situacional completo del entorno. Esta funcionalidad debería integrar indicadores como la presencia o ausencia de personas en la vivienda, utilizando sensores de movimiento, horarios habituales o incluso rutinas declaradas por el usuario. Igualmente, se sugiere incluir mecanismos que permitan detectar silencios prolongados e inusuales, los cuales podrían indicar caídas, pérdida de conciencia o situaciones de emergencia silenciosas, especialmente en usuarios con movilidad reducida.
  \item Ampliar los mecanismos de notificación mediante módulos GSM y servicios de notificaciones push: se recomienda diversificar e incrementar la robustez de los canales de envío de alertas mediante la incorporación de módulos GSM (Global System for Mobile Communications), los cuales permitirían el envío de mensajes SMS directamente desde el dispositivo sin necesidad de conexión a internet. Esto resulta especialmente útil en contextos con baja conectividad o en situaciones donde la red local pueda fallar durante una emergencia. De igual forma, se sugiere integrar notificaciones push móviles, a través de plataformas como Firebase Cloud Messaging (FCM), que permitirían alertas más inmediatas, personalizadas y con capacidad de interacción desde teléfonos inteligentes.
  \item Uso de ensamblajes de modelos: La combinación de múltiples modelos de detección, como CNNs y LSTMs, podría mejorar la precisión y robustez del sistema al aprovechar las fortalezas individuales de cada arquitectura.
  \item Desarrollo de un agente controlador: Un agente controlador que gestione las alertas de emergencia, priorizando y filtrando las notificaciones para evitar falsas alarmas y garantizar que las alertas críticas se envíen de manera oportuna.
\end{itemize}