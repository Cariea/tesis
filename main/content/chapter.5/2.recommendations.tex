\usection{Recomendaciones}

\begin{itemize}
  \item Incorporar modelos de aprendizaje secuencial más avanzados: La integración de arquitecturas como transformers adaptados al procesamiento de audio permitiría mejorar la comprensión del contexto acústico y reducir la incidencia de falsos positivos o negativos en escenarios ambiguos.
  \item Desarrollar interfaces más intuitivas y accesibles: Se sugiere implementar una interfaz de usuario que permita a los usuarios configurar umbrales de alerta, definir eventos críticos personalizados y gestionar contactos de emergencia de forma amigable, incluyendo opciones accesibles para personas mayores.
  \item Ampliar los mecanismos de notificación mediante módulos GSM y servicios de notificaciones push: se recomienda diversificar e incrementar la robustez de los canales de envío de alertas mediante la incorporación de módulos GSM (Global System for Mobile Communications), los cuales permitirían el envío de mensajes SMS directamente desde el dispositivo sin necesidad de conexión a internet. Esto resulta especialmente útil en contextos con baja conectividad o en situaciones donde la red local pueda fallar durante una emergencia. De igual forma, se sugiere integrar notificaciones push móviles, a través de plataformas como Firebase Cloud Messaging (FCM), que permitirían alertas más inmediatas, personalizadas y con capacidad de interacción desde teléfonos inteligentes.
  \item Uso de ensamblajes de modelos: La combinación de múltiples modelos de detección, como CNNs y LSTMs, podría mejorar la precisión y robustez del sistema al aprovechar las fortalezas individuales de cada arquitectura. De esta manera, se podrían mitigar las debilidades de un solo modelo y aumentar la capacidad de generalización ante diversos escenarios acústicos. 
  \item Sistema de retroalimentación del usuario: Implementar un mecanismo que permita a los usuarios confirmar o descartar alertas, ayudando al sistema a aprender y mejorar su precisión con el tiempo.
  \item Incorporar el uso de sensores de presencia como el C1001 mmWave Human Detection Sensor de Acconeer, que utiliza tecnología de radar para detectar la presencia y el movimiento de personas en un entorno. Pudiendo detectar caídas, movimientos inusuales o la ausencia de movimiento, ritmo cardíaco y respiración, sin la necesidad de estar fisicamente unido al cuerpo del usuario. Esto podría complementar el sistema de identificación de sonidos proporcionando una capa adicional de validacion de estado del usuario o una serie de parametros adicionales para el entrenamiento de modelos de aprendizaje automático.
\end{itemize}