\usection{Conclusiones}

El análisis conceptual de la analítica de sonidos ambientales estableció la necesidad fundamental de ir más allá de los enfoques de intensidad o modelado secuencial simple (ARIMA/Márkov). Se determinó que la base metodológica del sistema radica en la caracterización semántica del audio (YAMNet) y el modelado contextual de sus secuencias (LSTM Autoencoder), lo que permitió definir la estrategia de detección de anomalías.

El diseño del sistema proporcionó una arquitectura de microservicios en el borde (Edge AI) sobre una Raspberry Pi, demostrando la posibilidad de cumplir con los requisitos de privacidad (procesamiento local) y modularidad (RNF1 y RNF4). Esta estructura de seis capas garantiza un flujo de trabajo adaptable para integrar la clasificación acústica y el análisis de secuencias de manera segura y escalable.

La implementación del sistema culminó con la consecución de un prototipo funcional que integra el clasificador semántico con el modelo LSTM Autoencoder. Se integró un proceso robusto de preprocesamiento, incluyendo el manejo de variables cíclicas y la codificación One-Hot de los eventos, lo que habilitó al modelo para aprender con fidelidad los patrones de comportamiento normales de las secuencias cpn baja latencia.

La validación   del sistema confirmó la alta capacidad de detección del modelo LSTM Autoencoder mejorado, alcanzando un rendimiento de $Recall$ del 100 y $Precision$ superior al 90. Este resultado demuestra que el sistema es completamente funcional para la detección de emergencias, al eliminar por completo los Falsos Negativos ($FN=0$), lo que asegura que ninguna situación crítica sea omitida.

Finalmente, la elaboración de la documentación garantiza la continuidad y replicabilidad de la solución. El manual detalla la arquitectura modular y los procedimientos técnicos para la configuración del entorno de ejecución en el borde, lo que asegura el mantenimiento del sistema como una herramienta de seguridad no invasiva y de gran utilidad para personas vulnerables.
 
% y una $Precision$ superior al 90%, y lo más crítico, eliminando por completo los Falsos Negativos ($FN=0$). Esto valida que el enfoque de monitoreo acústico contextual es apto para la detección fiable de emergencias, al garantizar que ninguna situación crítica sea omitida.Finalmente, la elaboración de la documentación se centró en describir la arquitectura modular de seis capas y los procedimientos técnicos detallados para la configuración del entorno de ejecución en el borde, asegurando la sostenibilidad y el mantenimiento del sistema como una herramienta accesible y no invasiva para la seguridad de personas vulnerables.
% La revisión teórica y el análisis de enfoques previos permitieron establecer las bases conceptuales y tecnológicas para diseñar un sistema capaz de detectar patrones acústicos asociados a situaciones de riesgo. La integración de modelos como YAMNet y Vosk demostró que es posible realizar detección y clasificación de eventos sonoros en tiempo real desde dispositivos de bajo consumo, como Raspberry Pi.

% La arquitectura del sistema, basada en una red distribuida de micrófonos conectados a nodos de procesamiento local, permitió generar un perfil acústico del entorno, diferenciando entre sonidos rutinarios y eventos anómalos. La incorporación de cadenas de Márkov como modelo probabilístico inicial brindó un nivel de razonamiento temporal básico, aunque sus limitaciones estructurales derivaron en la exploración de arquitecturas más avanzadas, como redes LSTM, capaces de modelar secuencias acústicas de forma más coherente.

% Durante la validación, los modelos mostraron alta tasa de éxito en pruebas con eventos claramente definidos, como gritos, cristales rotos o sonidos fuertes, alcanzando niveles del 100\% de aciertos. Sin embargo, se identificaron desafíos en la detección contextualizada de palabras clave o en sonidos menos estructurados, lo que evidencia la necesidad de seguir mejorando la sensibilidad y especificidad del sistema en escenarios complejos.

% El desarrollo adoptó la metodología espiral, lo cual permitió iterar entre diseño, implementación y validación de forma progresiva. Esta estrategia fue clave para ajustar la solución a las restricciones del entorno real, abordando problemas como la gestión de ruido ambiental, las limitaciones de hardware y los requisitos legales sobre privacidad de datos.

% A nivel social y ético, el sistema representa una herramienta accesible y no invasiva, especialmente relevante para personas en situación de vulnerabilidad, como adultos mayores o personas con discapacidad. Al evitar el almacenamiento de datos sensibles y operar en tiempo real, se garantiza el cumplimiento con principios constitucionales de confidencialidad y autonomía.
