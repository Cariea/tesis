\usection{Conclusiones}

El análisis conceptual de la analítica de sonidos ambientales establece la necesidad fundamental de ir más allá de los enfoques de intensidad o modelado secuencial simple (ARIMA/Márkov). Se determina que la base metodológica del sistema radica en la caracterización semántica del audio (YAMNet) y el modelado contextual de sus secuencias (LSTM Autoencoder), lo que permite definir la estrategia de detección de anomalías.

El diseño del sistema proporciona una arquitectura híbrida de microservicios en el borde (Edge AI) sobre una Raspberry Pi, demostrando la posibilidad de cumplir con los requisitos de privacidad (procesamiento local) y modularidad. Esta estructura de seis capas garantiza la adaptabilidad del prototipo pudiendo cambiar componentes como el clasificador acústico o el modelo de detección de anomalías sin afectar la integridad del sistema.

La implementación del sistema culmina con un prototipo funcional que integra un clasificador de audio preentrenado (YAMNet) con el modelo LSTM Autoencoder. Se implementa un proceso robusto de preprocesamiento de datos, que incluye la representación cíclica de variables temporales y categóricas de los eventos, lo que permite al modelo aprender con fidelidad los patrones de comportamiento normales de las secuencias y operar con baja latencia.

La validación del sistema confirma la alta capacidad de detección del modelo LSTM Autoencoder mejorado, alcanzando un rendimiento de Recall del 100 y Precisión superior al 90. Este resultado demuestra que el sistema es completamente funcional para la detección de emergencias, al eliminar por completo los Falsos Negativos (FN = 0), lo que asegura que ninguna situación crítica sea omitida.

La adopción del Modelo Espiral como metodología de desarrollo permitie gestionar eficazmente la incertidumbre técnica inicial, facilitando iteraciones progresivas que evolucionaron desde enfoques simples basados en intensidad sonora hasta una arquitectura robusta de IA en el borde.

Finalmente, la elaboración de la documentación garantiza la continuidad y replicabilidad de la solución. El manual detalla la arquitectura modular y los procedimientos técnicos para la configuración del entorno de ejecución en el borde, lo que asegura el mantenimiento del sistema como una herramienta de seguridad no invasiva y de gran utilidad para personas vulnerables.