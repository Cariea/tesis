\usection{Justificación}

En la actualidad, la atención a personas en condiciones de vulnerabilidad requiere el desarrollo de soluciones tecnológicas que permitan una respuesta rápida ante situaciones críticas, sin comprometer la privacidad ni depender de una supervisión constante. En este contexto, el presente proyecto propone un sistema de monitoreo acústico basado en inteligencia artificial, diseñado para identificar sonidos inusuales que puedan estar asociados a eventos de emergencia. Al generar alertas automáticas en tiempo real, el sistema facilita la intervención oportuna de familiares, cuidadores o servicios de asistencia ante incidentes potenciales.

Para su desarrollo se aplicó una metodología de trabajo basada en el modelo de desarrollo en espiral, lo que permitió avanzar por etapas, realizar pruebas continuas y ajustar el diseño según los resultados obtenidos. Esta estrategia fue útil para integrar componentes de hardware y software y adaptarlos a distintos entornos de prueba. Se emplearon herramientas como la Raspberry Pi y modelos de inteligencia artificial.

El sistema fue implementado utilizando recursos disponibles localmente y orientado a realizar el procesamiento de datos de forma local, sin almacenar audios ni requerir conexión a servicios externos, resguardando así la privacidad del usuario.

Desde una perspectiva social, el sistema puede beneficiar a personas que, por condiciones físicas o emocionales, se encuentran en mayor riesgo de enfrentar emergencias sin posibilidad de pedir ayuda inmediata, como en casos de caídas, crisis de salud o situaciones de violencia doméstica. También contribuye a la labor de cuidadores y servicios de emergencia al ofrecer alertas tempranas.

En el ámbito institucional, el proyecto plantea una solución aplicable en hogares de cuidado, hospitales o comunidades, ya que su arquitectura de bajo costo y su implementación con componentes disponibles permiten su adaptación a contextos con recursos limitados. Esta característica puede ser relevante en la formulación de futuras políticas públicas que busquen integrar herramientas automatizadas en estrategias de atención ciudadana.

El sistema también se vincula con los Objetivos de Desarrollo Sostenible (ODS), particularmente el ODS 3: Salud y bienestar, al aportar un mecanismo de alerta temprana en situaciones críticas, y el ODS 10: Reducción de desigualdades, al ofrecer una solución técnica accesible que no depende de infraestructura compleja. Su diseño adaptable lo hace viable para diferentes contextos sociales, promoviendo mayor equidad en el acceso a herramientas de apoyo.

Por último, este trabajo aporta al ámbito académico mediante el desarrollo de un prototipo funcional con recursos locales, evidenciando la capacidad de la ingeniería venezolana para abordar problemáticas sociales mediante soluciones aplicadas. La documentación generada servirá como base para nuevas iniciativas en inteligencia artificial enfocadas en necesidades reales, incentivando así la investigación y la innovación en el país.
