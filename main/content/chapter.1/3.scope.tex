\usection{Alcance}

El proyecto tiene como objetivo desarrollar un sistema de monitoreo acústico que perfile el comportamiento de los habitantes de una vivienda, a través del reconocimiento y clasificación de los sonidos ambientales en las diferentes estancias de la misma. El sistema generará alarmas en tiempo real que alertarán a contactos de emergencia predefinidos. Para ello se llevará a cabo un análisis de los conceptos asociados a la analítica de sonidos donde se identificarán los conceptos necesarios para diseñar el sistema.

Con base en los resultados del análisis, se procede a diseñar el sistema de monitoreo acústico capaz de reconocer sonidos. El diseño comprende la definición de la arquitectura del sistema, contemplando los componentes esenciales, como la red de micrófonos para la captura de sonidos y los mecanismos de alerta, detección de anomalías y de reconocimiento acústico. El diseño considera la importancia de la privacidad de los usuarios. Según los resultados del análisis realizado para el desarrollo del sistema de monitoreo acústico, se evalúa si es necesario crear un dataset desde cero, obtener uno de internet y modificarlo según sea necesario, o bien emplear un modelo ya preentrenado.

Una vez completado el diseño, se procede a la implementación del sistema de monitoreo acústico. Esta fase incluye la configuración e instalación de los componentes físicos, como los micrófonos distribuidos en las estancias de la vivienda, asegurando una cobertura adecuada para la captura de sonidos relevantes. Los algoritmos de procesamiento de señales y los modelos de inteligencia artificial, definidos en la fase de diseño, fueron desarrollados y adaptados para realizar el reconocimiento y clasificación de los sonidos.

Se desarrollaron los mecanismos de alerta y notificación, que emiten avisos a los contactos de emergencia predefinidos.

Luego se realiza la validación del sistema, se lleva a cabo pruebas funcionales en ambientes controlados y no controlados para comparar los resultados obtenidos con los objetivos planteados y realizar ajustes en caso de ser necesario.

Finalmente se elabora la documentación del sistema, manuales de usuario, manuales de sistema, descripción de los componentes, diagramas y cualquier otro documento necesario.
