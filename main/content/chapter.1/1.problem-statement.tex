\usection{Planteamiento del Problema}

\cite{alpha_cephei_vosk_2024}
\cite{apple_use_2023}
\cite{asamblea_nacional_constituyente_constitucion_1999}
\cite{bobadilla_cadenas_2010}
\cite{boehm_spiral_1988}
\cite{carmona_enchufe_2024}
\cite{covenin_norma_2005}
\cite{dolui_comparison_2017}
\cite{dubs_estrategia_2004}
\cite{garcia_ciencia_2018}
\cite{goldsborough_tour_2016}
\cite{hernandez_metodologia_2014}
\cite{hurtado_metodologia_2000}
\cite{hyndman_forecasting_2018}
\cite{igba_analysing_2016}
\cite{kaggle_yamnet_2024}
\cite{malmberg_real-time_2021}
\cite{matas_introduccion_2024}
\cite{medina_edge_2019}
\cite{mondragon_clasificacion_2021}
\cite{naciones_unidas_envejecimiento_2022}
\cite{naciones_unidas_transformar_2015}
\cite{pressman_ingenieria_2010}
\cite{provost_data_2013}
\cite{ralla_hallan_2024}
\cite{republica_bolivariana_venezuela_ley_2001}
\cite{republica_bolivariana_venezuela_ley_mensajes_2001}
\cite{republica_bolivariana_venezuela_ley_organica_2010}
\cite{richardson_getting_2016}
\cite{russell_artificial_2022}
\cite{sabo_nestjs_2020}
\cite{shi_edge_2016}
\cite{tamayo_proceso_2004}
\cite{torija_metodologia_2018}
\cite{upel_manual_2004}
\cite{upton_raspberry_2020}
\cite{yuill_python_2006}


En la actualidad, existe un número significativo de casos de personas que sufrieron algún accidente o complicación de una condición de salud, producto de la detección tardía de estos eventos, una tendencia que ha ido en aumento debido al envejecimiento de la población, que de acuerdo a los expuesto por la \citeauthor{cepal_2005} \citeyear{cepal_2005}, dice que: “el envejecimiento demográfico en América Latina se triplicará entre el 2000 y 2050, con un aumento significativo en la proporción de personas mayores de 60 años” y a la estructura familiar moderna, que según \cite{pena2016reciente}, ``El aumento de los hogares unipersonales se ha convertido en un fenómeno extendido dentro del mundo occidental desarrollado''. Esta situación, expone a las personas a múltiples riesgos, especialmente cuando se enfrentan a emergencias domésticas, problemas de salud o accidentes que comprometen su bienestar físico y emocional. La falta de asistencia inmediata en situaciones críticas puede transformar un incidente inicialmente manejable en una amenaza grave para la vida.

Algunos casos ilustran estos riesgos. Un ejemplo es el de un hombre octogenario en Zaragoza, España, cuyo fallecimiento en su domicilio pasó desapercibido durante una semana. No fue hasta que una vecina notó su ausencia que se alertó a las autoridades. Aunque la causa de su muerte fue natural, este caso pone en evidencia la vulnerabilidad de las personas que viven solas y la falta de un mecanismo eficiente que pueda detectar la inactividad prolongada o cambios abruptos en la rutina diaria (Ralla 2024). En una situación como esta, el problema no es la atención médica, sino la ausencia de una alerta automática que notifique una anomalía grave (la total falta de actividad) a familiares o servicios de emergencia. De forma similar, en un caso no fatal como una caída, la falta de esa misma alerta inmediata sí puede derivar en graves complicaciones físicas, transformando un accidente manejable en una crisis de salud. Estos incidentes no son excepcionales, sino que representan una problemática real, especialmente en el contexto de los hogares unipersonales.

El envejecimiento de la población es un fenómeno natural que está incrementando la cantidad de personas mayores que viven en solitario. De acuerdo a lo comentado por la Organización de Naciones Unidas (2022): ``se estima que para el 2050, el 16\% de la población mundial estará compuesta por personas mayores de 65 años, muchas de las cuales optarán o se verán forzadas a vivir sin compañía''. Esta tendencia plantea un desafío crítico en cuanto a su seguridad y bienestar. La falta de asistencia oportuna en emergencias no solo representa una vulnerabilidad significativa para la vida de estas personas ante eventos que pongan en riesgo su vida o integridad física, sino que, a largo plazo, puede contribuir a un deterioro progresivo de su estado de salud y autonomía. Además, situaciones como caídas, ataques cardíacos o accidentes domésticos, que podrían ser atendidos a tiempo, se convierten en amenazas letales debido a la ausencia de un sistema de detección temprana.

Aunque en los últimos años se han desarrollado algunas soluciones tecnológicas para abordar esta problemática, como el dispositivo Zoe Fall, que detecta caídas a través de ondas WiFi, o el Apple Watch, que incluye una función de detección de caídas, estas tecnologías no son accesibles para todos ni cubren una gama suficientemente amplia de emergencias. De acuerdo a lo comentado por José Antonio Carmona (2024): ``estas herramientas están mayormente enfocadas en eventos puntuales, como caídas, pero no proporcionan un monitoreo continuo y adaptable a otras situaciones que podrían requerir detección temprana''. Adicionalmente, la mayoría de estos dispositivos dependen de la capacidad del usuario para activarlos o configurarlos adecuadamente, lo cual no siempre es viable en personas que, por su edad o condición, pueden no estar familiarizadas con el uso de la tecnología.

La falta de mecanismos efectivos para detectar emergencias a tiempo no solo prolongará el sufrimiento de las personas en situaciones críticas, sino que también incrementará las tasas de mortalidad y discapacidad asociadas a estos eventos. La inacción en este sentido implicaría ignorar las necesidades de una población vulnerable, cuya seguridad y calidad de vida pueden verse directamente influenciadas por la imposibilidad de notificar estos eventos.

Ante este panorama, se evidencia la necesidad de desarrollar una solución tecnológica que supere las limitaciones de los dispositivos actuales, ofreciendo un monitoreo más integral, no invasivo y accesible. Por lo tanto, este trabajo se centra en investigar y proponer un sistema de monitoreo inteligente basado en el análisis del entorno acústico. La pregunta central que guía esta investigación es: ¿Es posible, utilizando inteligencia artificial en dispositivos de bajo costo, crear un sistema que aprenda el comportamiento sonoro habitual de un hogar y genere alertas fiables ante patrones anómalos, respetando al mismo tiempo la privacidad del usuario?

Este sistema, ofrecerá un monitoreo continuo, creando un perfil acústico de cada ambiente del hogar que permitirá diferenciar entre sonidos típicos y atípicos en su propio entorno.

Al detectar un sonido que se salga de los patrones normales, el sistema enviará una consulta al individuo para confirmar si está en buen estado; en caso de no recibir respuesta o de que se confirme una situación de emergencia, enviará una alerta inmediata a los contactos de emergencia predefinidos

El sistema monitorea el sonido en el ambiente y aprende de los patrones habituales, identificando qué ruidos son normales y cuáles no, así como los momentos en que ocurren. Esto permite generar un perfil de audio adaptado a cada entorno del hogar, ya sea de una persona mayor, una persona con discapacidad, un hogar con muchos niños, o un entorno con vecinos que escuchan música alta con frecuencia. Este perfil facilita la detección de situaciones anómalas que puedan representar un riesgo, especialmente en viviendas donde viven personas solas, quienes pueden enfrentar desafíos por la falta de compañía o asistencia. Así, el sistema está diseñado no solo para adultos mayores o personas con discapacidad, sino para cualquier persona en situación de vulnerabilidad.

Esta solución tiene el potencial de reducir los riesgos asociados a la detección tardía de estos eventos, proporcionando a las personas un mayor nivel de seguridad sin comprometer su autonomía.

