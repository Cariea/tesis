\usection{Tipo de Investigación}

El presente trabajo constituye una investigación de tipo proyectiva. De acuerdo con \citeauthor{hurtado_metodologia_2000} \citeyear{hurtado_metodologia_2000}, la investigación proyectiva tiene como objetivo diseñar o crear propuestas dirigidas a resolver determinadas situaciones. Los proyectos de arquitectura e ingeniería, el diseño de maquinarias, la creación de programas de intervención social, el diseño de programas de estudio, los inventos, la elaboración de programas informáticos, entre otros, siempre que estén sustentados en un proceso de investigación, son ejemplos de investigación proyectiva. Este tipo de investigación potencia el desarrollo tecnológico (p. 325).

Basado en esta definición, el desarrollo de un sistema de monitoreo acústico para identificar sonidos y generar alertas de emergencia se adapta a este tipo de investigación. El sistema propuesto en este trabajo se fundamenta en un proceso de investigación que incluye la revisión documental de conceptos como edge computing, el uso de modelos pre-entrenados para la clasificación de audio y el análisis de secuencias temporales para la detección de anomalías, así como el diseño e implementación de un prototipo funcional. Esto cumple con los requisitos de una investigación proyectiva, ya que no solo se analiza y diagnostica el problema, sino que también se desarrolla una solución tecnológica aplicable en entornos reales.

