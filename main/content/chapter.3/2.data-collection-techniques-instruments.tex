\usection{Técnicas e Instrumentos de Recolección de Datos}

Según Hernández, Fernández y Baptista (2014), las técnicas de recolección de datos son procedimientos sistemáticos que permiten obtener información relevante para la investigación, mientras que los instrumentos son las herramientas específicas utilizadas para registrar dicha información.

En este estudio, se emplearon técnicas tanto documentales como experimentales. La técnica documental se utilizó para recopilar información teórica y antecedentes relacionados con los temas de la investigación. Para ello, se realizó una revisión bibliográfica de fuentes como artículos científicos, tesis y normativas legales, con el objetivo de construir el marco teórico que sustenta el proyecto. Esta técnica permitió fundamentar el diseño del sistema y justificar las decisiones técnicas adoptadas. Por otro lado, la técnica experimental se aplicó durante la fase de implementación y validación del sistema. Para ello, se diseñaron experimentos controlados en los que se capturaron y analizaron sonidos ambientales utilizando micrófonos y dispositivos Raspberry Pi. Los instrumentos utilizados en esta fase incluyeron software de procesamiento, así como protocolos de registro de datos para documentar los resultados de las pruebas. Como señala Tamayo (2004), "la recolección de datos debe ser precisa y sistemática para garantizar que la información obtenida sea válida y confiable" (p. 145). En este sentido, se implementaron protocolos estandarizados para la captura y clasificación de sonidos, asegurando que los datos fueran consistentes y representativos de los escenarios reales en los que se espera que opere el sistema.
