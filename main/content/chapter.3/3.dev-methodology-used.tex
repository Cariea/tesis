\usection{Metodología de Desarrollo Utilizada}

Dado que al inicio del proyecto no se contaba con una definición completa de los requisitos, se necesitaba una metodología que fuera flexible y evolutiva. La elección del modelo de desarrollo se basó en la necesidad de gestionar activamente los riesgos técnicos, permitir una mejora continua a través de iteraciones y adaptar el sistema a los desafíos que surgieran durante la implementación.

Esto implicaba ciclos de retroalimentación constantes, validación continua y refinamiento iterativo del sistema, características esenciales para alcanzar una solución en un contexto tan dinámico. La elección de una metodología de desarrollo adaptable permitió no solo enfrentar la incertidumbre técnica, sino también ajustar oportunamente el diseño según los aprendizajes obtenidos en cada etapa del proceso. Con esto en mente comparamos las distintas metodologías basándonos en esas necesidades

% TODO - Agregar Tabla

Aunque metodologías ágiles como Scrum también cumplen con estos criterios, se optó por el modelo Espiral debido a su énfasis explícito en el 'Análisis de Riesgos' como una fase formal en cada ciclo. Dada la incertidumbre técnica del proyecto, esta característica fue considerada fundamental para mitigar de manera proactiva cualquier desafío técnico o de gestión que pudiera surgir.

Según lo propuesto por Barry Boehm en 1986, es un enfoque iterativo y flexible que combina elementos del desarrollo incremental y el prototipado, permitiendo gestionar riesgos y adaptar el sistema a medida que avanza el proyecto. Este modelo ha sido ampliamente reconocido por su capacidad para gestionar el desarrollo de sistemas complejos, ya que combina la naturaleza iterativa del prototipado con la estructura sistemática del modelo en cascada (Pressman, 2014, p. 39). En cada iteración, se pueden realizar ajustes en el plan del proyecto, permitiendo adaptar el software a las necesidades emergentes y reducir riesgos antes de que se conviertan en problemas críticos (Pressman, 2014, p. 40).

Según Boehm (1988), el Modelo Espiral se caracteriza por su estructura cíclica, en la que cada iteración incluye cuatro fases principales:

\begin{enumerate}
  \item \textbf{Planificación.} donde se identifican los objetivos del proyecto, los requisitos principales y las restricciones que pueden influir en su desarrollo (Boehm, 1988). Se consideran alternativas y estrategias para alcanzar estos objetivos de manera eficiente.
  \item \textbf{Análisis de riesgos.} Se identifican y evalúan los riesgos potenciales. Se definen acciones para reducir los riesgos identificados y se evalúan alternativas existentes partiendo de prototipos, simulaciones y softwares de análisis. En este ciclo, existen varios prototipos como plantillas de diseño o componentes funcionales.
  \item \textbf{Desarrollo.} Una vez que los riesgos han sido evaluados, se llevan a cabo las actividades de ingeniería de software. Esta fase incluye el diseño detallado, la codificación del sistema y las pruebas del producto, desarrollando el incremento correspondiente a la iteración actual. (Boehm, 1988).
  \item \textbf{Evaluación.} se revisan los productos obtenidos en la iteración actual. Esta revisión implica la participación de los interesados para verificar que el sistema cumple con los requisitos establecidos y permite realizar ajustes para la siguiente iteración. Esta fase es clave para garantizar la mejora continua del software y la adaptación a nuevas necesidades (Pressman, 2014, p. 40).
\end{enumerate}

Como señala Boehm (1988), "el Modelo Espiral es particularmente adecuado para proyectos con altos niveles de incertidumbre y requisitos cambiantes, ya que permite incorporar retroalimentación continua y adaptar el diseño en función de los resultados obtenidos”. En este trabajo, esta metodología permitió gestionar los riesgos técnicos y asegurar que el sistema cumpliera con los objetivos planteados.

% TODO - Agregar Tabla
