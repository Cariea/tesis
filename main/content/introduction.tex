\phantomsection
\chapter*{Introducción}
\addcontentsline{toc}{chapter}{Introducción}

Este trabajo propone el desarrollo de un sistema distribuido que, mediante el uso de inteligencia artificial, permite la detección de anomalías acústicas como herramienta de apoyo para la alerta temprana ante posibles emergencias. El sistema está enfocado en servir no solo a personas mayores o con discapacidades, sino también a cualquier individuo que se encuentre en una condición de vulnerabilidad. Su funcionamiento se basa en dispositivos que primero clasifican los sonidos ambientales en tiempo real para, posteriormente, analizar la secuencia de dichas clasificaciones. Este análisis permite al sistema aprender el comportamiento acústico ``normal'' de un entorno y, en consecuencia, identificar eventos que se desvían de esa normalidad para generar una alerta a contactos predefinidos. La importancia de este proyecto radica en su capacidad de prevenir que se agraven las consecuencias ante un evento que comprometa la salud de un individuo si no se detectan a tiempo, ofreciendo un monitoreo no invasivo que respeta la privacidad. Además, beneficia a familias, cuidadores, instituciones públicas y privadas y facilita una intervención más oportuna de los servicios de emergencia, incluso en casos de violencia doméstica o el monitoreo de personas con depresión. El propósito es crear un entorno más seguro mediante la identificación de situaciones anormales y alertar a quienes puedan ofrecer ayuda, contribuyendo a una mejor calidad de vida al facilitar una respuesta más oportuna ante emergencias.

Para el desarrollo del sistema, se empleó una metodología basada en el modelo de desarrollo en espiral, que permite un enfoque iterativo y flexible. Cada ciclo del proceso abarca fases de planificación, análisis de riesgos, desarrollo y evaluación, permitiendo ajustes continuos en el diseño y la implementación del sistema. Esta metodología es adecuada para proyectos con incertidumbre, ya que facilita la evolución del sistema según la retroalimentación obtenida, asegurando su mejora progresiva a lo largo del desarrollo.

El presente trabajo está estructurado en cinco capítulos, En el Capítulo I, referido al Planteamiento del Problema, se describe la problemática, se establecen los objetivos, justificación, alcance y limitaciones del proyecto. En el Capítulo II, se expone el Marco Teórico donde se recopilan los antecedentes y las bases teóricas que sustentan el trabajo. En el Capítulo III, se presenta el Marco Metodológico, que describe el tipo de investigación, técnicas e instrumentos de recolección de datos, metodología de desarrollo y el procedimiento metodológico. En el Capítulo IV, se expone el Desarrollo y Resultados, donde se describe como el procedimiento metodológico dio respuesta a cada uno de los objetivos planteados, y en el Capítulo V, se presentan las conclusiones y recomendaciones sobre el trabajo realizado. Finalmente se listan las referencias bibliográficas utilizadas, seguidas de los anexos y apéndices.

\clearpage
