\phantomsection
\chapter*{Introducción}
\addcontentsline{toc}{chapter}{Introducción}

Actualmente, el desarrollo de sistemas de alerta temprana se ha convertido en una prioridad social y tecnológica debido a la detección tardía de accidentes o complicaciones de salud en individuos vulnerables que viven o pasan largos períodos de tiempo solos. Esta problemática se agudiza por el envejecimiento demográfico, la CEPAL proyecta que la población mayor de 60 años en América Latina se triplicará entre 2000 y 2050, y por la estructura familiar moderna, marcada por el aumento de hogares unipersonales. Esta confluencia de factores expone a las personas a múltiples riesgos, ya que la falta de asistencia inmediata en situaciones críticas puede transformar un incidente inicialmente manejable en una amenaza grave para la vida, incrementando las tasas de mortalidad y discapacidad asociadas.

Si bien existen soluciones tecnológicas, como dispositivos de detección de caídas integrados en wearables (e.g., Apple Watch) o dispositivos de alerta dedicados (collares o botones de emergencia), estas presentan limitaciones en cuanto a accesibilidad y cobertura. Su enfoque se centra en eventos puntuales y no proveen un monitoreo continuo y adaptable a diversas emergencias, siendo particularmente ineficaces si la persona se encuentra inconsciente o es sometida a coacción. Adicionalmente, su dependencia de la activación manual constituye una barrera tecnológica significativa para la población de edad avanzada o con condiciones limitantes.

Este trabajo aborda la necesidad de mecanismos efectivos de alerta temprana mediante el desarrollo de un sistema distribuido que utiliza Inteligencia Artificial para la detección de anomalías acústicas. El sistema está diseñado para servir a cualquier individuo en condición de vulnerabilidad. Su arquitectura funcional emplea una Raspberry Pi operando como sistema embebido para la captura y procesamiento inicial en el borde (Edge Computing), complementada por un servidor central que gestiona las alertas y comunicaciones externas, manteniendo el dispositivo de procesamiento de audio completamente aislado de internet. Se diseña una arquitectura de software que utiliza modelos preentrenados (YAMNet para detección de sonidos ambientales y Vosk para clasificación de palabras clave). Posteriormente, se aplican modelos avanzados de aprendizaje automático (Isolation Forest y LSTM) para el descubrimiento y clasificación de secuencias de patrones anómalos a partir de dichas detecciones, lo que permite al sistema caracterizar el comportamiento acústico normal del entorno e identificar desviaciones significativas que podrían indicar situaciones de emergencia.

Para la implementación, se emplea una metodología basada en el Modelo de Desarrollo en Espiral, un enfoque iterativo adecuado para proyectos con alta incertidumbre técnica. Esta metodología permitió un desarrollo efectivo mediante la iteración de las fases de Planificación, Análisis y mitigación de riesgos, Desarrollo incremental y evaluación, y Planificación de la siguiente iteración. La solución obtenida no solo cumple con los requerimientos funcionales establecidos, sino que garantiza la privacidad de los usuarios al realizar el procesamiento primario de los datos de audio de manera local. La importancia de este proyecto radica en su potencial para beneficiar a familias, cuidadores e instituciones, facilitando una intervención más oportuna ante diversas emergencias, incluyendo casos de violencia doméstica o el monitoreo de personas con depresión, contribuyendo así a una mejor calidad de vida.

El presente trabajo está estructurado en cinco capítulos: el Capítulo I describe el Planteamiento del Problema, donde se expone la problemática, los objetivos, justificación, alcance y limitaciones del proyecto; el Capítulo II expone el Marco Teórico, que recopila los antecedentes y bases teóricas que sustentan el trabajo; el Capítulo III presenta el Marco Metodológico, describiendo el tipo de investigación, técnicas e instrumentos de recolección de datos, metodología de desarrollo y procedimiento metodológico; el Capítulo IV detalla el Desarrollo y Resultados, exponiendo cómo el procedimiento metodológico dio respuesta a cada objetivo planteado; y el Capítulo V contiene las Conclusiones y Recomendaciones sobre el trabajo realizado. Finalmente, se listan las referencias bibliográficas, seguidas de los anexos y apéndices.

\clearpage
