% Keywords command
\providecommand{\keywords}[1]
{
  \small
  \textbf{\textit{Palabras clave---}} #1
}

\begin{abstract}
  Este trabajo presenta el desarrollo de un sistema de monitoreo acústico basado en inteligencia artificial, orientado a clasificar sonidos ambientales y generar alertas de emergencia al identificar eventos anómalos en tiempo real que podrían implicar que individuos se encuentren vulnerables. La creciente adopción de tecnologías para el hogar inteligente crea el ecosistema ideal para desarrollar soluciones avanzadas de monitoreo no invasivo. En este marco, el proyecto aborda la detección de eventos críticos, como cambios en la rutina o gritos de auxilio, con el fin de ofrecer una herramienta de alerta temprana ante situaciones de vulnerabilidad que, en el pasado, han derivado en tragedias por falta de detección. El sistema fue desarrollado utilizando dispositivos Edge como Raspberry Pi, integrando modelos preentrenados como YAMNet para clasificación de sonidos y Vosk para detección de palabras clave con modelos propios que perfilan el comportamiento acústico con Isolation Forest, LSTM y Transformers. La metodología empleada está basada en el modelo de desarrollo en espiral, lo cual permitió ciclos iterativos de análisis, diseño, implementación y validación, adaptándose a desafíos técnicos como la variabilidad del entorno sonoro y las limitaciones de hardware. En el desarrollo, se exploraron distintos enfoques: se evaluaron modelos de series temporales como ARIMA y Prophet, así como cadenas de Markov para analizar anomalías basadas en cambios de estado. Se determinó que estos modelos, si bien podían identificar ciertos eventos puntuales, carecían de la robustez necesaria para modelar la complejidad del entorno acústico en tiempo real. Los resultados evidencian que el sistema es capaz de identificar patrones anómalos de sonido en tiempo real y generar notificaciones automáticas a contactos de emergencia. La solución, cumple con normas legales sobre privacidad al no almacenar audios y procesar todo localmente. Se concluye que este sistema representa una herramienta ética para el monitoreo de entornos domésticos, alineándose con los Objetivos de Desarrollo Sostenible en salud, bienestar e inclusión tecnológica.
\end{abstract}
\keywords{alertas de emergencia, dispositivos Edge, inteligencia artificial, monitoreo acústico, Raspberry Pi.}
