\providecommand{\keywords}[1]
{
  \small
  \textbf{\textit{Palabras clave---}} #1
}

\phantomsection
\addcontentsline{toc}{chapter}{Resumen}
\setcounter{page}{\value{abstractpage}}
\begin{center}
  \begin{figure}[h]
    \centering
    \includegraphics[width=0.5\textwidth]{ucab-logo.png} \\
    \centering
    \renewcommand{\baselinestretch}{1.5}
    \membrete
    {
      \textbf{\titulo}
    }
  \end{figure}
  \begin{table}[h!]
    \onehalfspacing
    \raggedleft
    \begin{tabular}{r l}
      Autor:           & Naim, Carmelo  \\
                       & Sotillo, César \\
      Tutor Académico: & \tutor         \\
      Fecha:           & Marzo, 2025    \\
    \end{tabular}
  \end{table}
  \textbf{Resumen} \\
\end{center}

Este trabajo presenta el desarrollo de un sistema de monitoreo acústico basado en inteligencia artificial, que clasifica sonidos ambientales y genera alertas de emergencia al identificar eventos anómalos en tiempo real que podrían implicar que individuos se encuentren en una posición de vulnerabilidad. En este marco, el proyecto aborda la detección de eventos críticos, como cambios en la rutina o gritos de auxilio, con el fin de ofrecer una herramienta de detección que pueda alertar de forma temprana sobre situaciones anormales que podrían indicar una posición de vulnerabilidad del individuo. El sistema fue desarrollado utilizando dispositivos Edge como Raspberry Pi, integrando modelos preentrenados como YAMNet para clasificación de sonidos y Vosk para detección de palabras clave con modelos propios que perfilan el comportamiento acústico con Isolation Forest, LSTM y Transformers. La metodología empleada está basada en el modelo de desarrollo en espiral, lo cual permitió ciclos iterativos de análisis, diseño, implementación y validación, adaptándose a desafíos técnicos como la variabilidad del entorno sonoro y las limitaciones de hardware. Los resultados evidencian que el sistema es capaz de identificar patrones anómalos de sonido en tiempo real y generar notificaciones automáticas a contactos de emergencia. La solución considera la privacidad del usuario al no almacenar audios y procesar todo localmente. Se concluye que este sistema representa una herramienta ética para el monitoreo de entornos domésticos, alineándose con los Objetivos de Desarrollo Sostenible en salud, bienestar e inclusión tecnológica.

\keywords{alertas de emergencia, dispositivos Edge, inteligencia artificial, monitoreo acústico, Raspberry Pi.}
