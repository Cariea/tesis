\usepackage[spanish,es-lcroman]{babel}
\usepackage[utf8]{inputenc} 
\usepackage[T1]{fontenc}
\usepackage[letterpaper, margin=1in]{geometry}
\usepackage{graphicx}
\graphicspath{{images/}}
\usepackage{newtxtext,newtxmath}
\usepackage{setspace}
\linespread{1.5}
%\setstretch{1.5}
\usepackage{tabularx}
\usepackage[colorlinks=true, linkcolor=black, citecolor=black, urlcolor=black]{hyperref}
\usepackage{url}
\usepackage{apacite}
\bibliographystyle{apacite}

\usepackage{lipsum} % For placeholder text

% Formato para los capítulos
\usepackage{titlesec}
\titlespacing{\chapter}{0pt}{0pt}{40pt}

% Formato para evitar la separación de palabras 
\tolerance=1
\emergencystretch=\maxdimen
\hyphenpenalty=10000
\hbadness=10000

% Formato de números de páginas
\usepackage{fancyhdr}
\fancyhf{}
\setlength{\headheight}{15pt}

\fancypagestyle{prelude}{
  \fancyhf{}
  \fancyfoot[C]{\roman{page}}
  \renewcommand{\headrulewidth}{0pt}
  \renewcommand{\footrulewidth}{0pt}
}

\fancypagestyle{chapterstyle}{
  \fancyhf{}
  \fancyhead[R]{\thepage}   % número arriba a la derecha
  \renewcommand{\headrulewidth}{0pt}
}

% Estilo para índices (número abajo centrado)
\fancypagestyle{plain}{
  \fancyhf{}
  \fancyfoot[C]{\thepage}
  \renewcommand{\headrulewidth}{0pt}
}

\renewcommand{\headrulewidth}{0pt}
%\renewcommand{\thepage}{\roman{page}}
\renewcommand{\thechapter}{\Roman{chapter}}

% Formato de sangrías y listas
\usepackage{enumitem}
\renewcommand{\labelitemi}{$\bullet$}
\setlength{\parindent}{1.25cm}
\setlist[itemize,1]{leftmargin=0.55in}
\setlist[itemize]{leftmargin=0.2in}

% Formato de la tabla de contenido
\usepackage{tocloft}
\addto\captionsspanish{
  \renewcommand{\contentsname}{Índice de Contenido}
  \renewcommand{\listfigurename}{Índice de Figuras}
  \renewcommand{\listtablename}{Índice de Tablas}
}
\renewcommand{\cftloftitlefont}{\hfill\normalsize\bfseries}
\renewcommand{\cftafterloftitle}{\hfill}

\renewcommand{\cftlottitlefont}{\hfill\normalsize\bfseries}
\renewcommand{\cftafterlottitle}{\hfill}

\renewcommand{\cfttoctitlefont}{\hfil\normalsize\bfseries}
\renewcommand{\cftaftertoctitle}{\hfil}

\setlength{\cftbeforetoctitleskip}{-20pt}
\setlength{\cftbeforeloftitleskip}{-20pt}
\setlength{\cftbeforelottitleskip}{-20pt}
\setcounter{secnumdepth}{4}
\setcounter{tocdepth}{4}

% TOC Chapter Style
%\cftpagenumbersoff{chapter}
\setlength{\cftbeforechapskip}{5pt}

% TOC Sections Style
\setlength{\cftsecindent}{0pt}
\setlength{\cftsubsecindent}{1.5em}
\setlength{\cftsubsubsecindent}{3em}
\setlength{\cftparaindent}{4.5em}

% Formato de los diferentes niveles de secciones
% \newcommand{\upper}[1]{\uppercase{#1}}
\usepackage{titlesec}
\titleformat{\chapter}
{\normalfont\normalsize\bfseries\centering}{}{0.3em}{}[\thispagestyle{empty}]
\titleformat{\section}
{\normalfont\normalsize\bfseries}{\thesection}{1em}{}
\titleformat{\subsection}
{\normalfont\normalsize\bfseries\narrower}{\thesubsection}{1em}{}
\titleformat{\subsubsection}
{\normalfont\normalsize\bfseries\itshape\narrower}{\thesubsubsection}{1em}{}
\titleformat{\paragraph}
{\normalfont\normalsize\itshape\narrower}{\theparagraph}{1em}{}

% Espaciado vertical ajustado
\titlespacing{\chapter}{0pt}{0pt}{18pt}
\titlespacing{\section}{0pt}{12pt}{6pt}
\titlespacing{\subsection}{0pt}{10pt}{5pt}
\titlespacing{\subsubsection}{0pt}{8pt}{4pt}
\titlespacing{\paragraph}{0pt}{6pt}{3pt}

% TOC Chapter Style
\renewcommand{\cftchapleader}{\cftdotfill{\cftdotsep}}
\setlength{\cftbeforechapskip}{1pt}

\makeatletter
\renewcommand{\l@chapter}[2]{%
  \ifnum \c@tocdepth >\m@ne
    \addpenalty{-\@highpenalty}%
    \noindent #1 \dotfill #2\par
  \fi}
\makeatother
                   
% Comandos para crear secciones no numeradas con entrada en el índice
\newcommand{\uchapter}[1]{%
  \addtocounter{chapter}{1}%
  \phantomsection
  \chapter*{\textbf{\setstretch{1.5}\selectfont Capítulo \thechapter \break #1}}%
  \addcontentsline{toc}{chapter}{\textbf{Capítulo \thechapter. #1}}%
}

\newcommand{\uextra}[2]{
  \addtocounter{chapter}{1}
  \phantomsection
  \chapter*{\centering\textbf{#1 \thechapter. #2}}
  \addcontentsline{toc}{chapter}{#1 \thechapter. #2}
  \thispagestyle{chapterstyle}
}


\newcommand{\usection}[1]{
  \phantomsection
  \section*{#1}
  \addcontentsline{toc}{section}{#1}
}
\newcommand{\usubsection}[1]{
  \phantomsection
  \subsection*{#1.}
  \addcontentsline{toc}{subsection}{#1}
}
\newcommand{\usubsubsection}[1]{
  \phantomsection
  \subsubsection*{#1.}
  \addcontentsline{toc}{subsubsection}{#1}
}
\newcommand{\uparagraph}[1]{
  \phantomsection
  \paragraph*{#1.}
  \addcontentsline{toc}{paragraph}{#1}
}

%auri
% Citas largas
    \long\def\citalarga#1{
      \medskip
      \begingroup
      \parindent 0pt
      \leftskip=1.25cm
      \rightskip=1.25cm
      #1 \par \medskip
      \endgroup
    }
% Formato personalizado para el índice de tablas
    \renewcommand{\cfttabpresnum}{Tabla\ }     % Prefijo antes del número
    \renewcommand{\cfttabaftersnum}{.\ }       % Punto después del número
    \renewcommand{\cfttabnumwidth}{6.5em}      % Espacio reservado para el número
% Formato de tablas
    \usepackage{caption}
    \captionsetup[table]{
      labelsep=newline,
      justification=justified,
      singlelinecheck=false,
      textfont=it,
      name=Tabla,
      font=small
    }
    \renewcommand{\thetable}{\arabic{table}}
% Formato personalizado para el índice de figuras





    \renewcommand{\cftfigpresnum}{Figura\ }    % Prefijo antes del número
    \renewcommand{\cftfigaftersnum}{.\ }       % Punto después del número
    \renewcommand{\cftfignumwidth}{6.5em}      % Espacio reservado para el número

    
% Formato de figuras
    \usepackage{caption}
    \captionsetup[figure]{
        labelfont={bf,it}, 
        justification=justified,
        singlelinecheck=false,
        labelsep=period,
        name=Figura,
        font=small
    }
    \renewcommand{\thefigure}{\arabic{figure}}
% Hipervínculo
    % \makeatletter
    % \def\url@leostyle{%
    %   \@ifundefined{selectfont}{\def\UrlFont{\rm}}{\def\UrlFont{\rmfamily}}}
    % \makeatother
    % \urlstyle{leo}
% Por si quieres que todas las figuras pendientes se impriman antes de seguir
    \usepackage{placeins}
%appendices
    \usepackage{longtable}
    \usepackage{booktabs}
    \usepackage{float}
%para que las fig/tab salgan despues de su definicion (pryuba a borrarlo luego ngl)
    \usepackage{flafter}
%Para indice de formulas (sadge)
   \usepackage{amsmath}
    \renewcommand{\theequation}{\arabic{equation}}

    \newcommand{\listequationsname}{Índice de Fórmulas}
    \newlistof{myequations}{equ}{\listequationsname}

    % Formato personalizado para el índice de fórmulas
    \renewcommand{\cftmyequationspresnum}{Ecuación\ }     % Prefijo antes del número
    \renewcommand{\cftmyequationsaftersnum}{.\ }         % Punto después del número
    \renewcommand{\cftmyequationsnumwidth}{5em}          % Espacio reservado para el número
    
    \newcommand{\myequation}[3]{%
      \refstepcounter{equation} % Avanza el contador sin mostrar número
      \begin{equation*}
        #1
      \end{equation*}
      \addcontentsline{equ}{myequations}{\protect\numberline{\theequation}#2}
      \vspace{-0.5em}
      {\fontsize{11pt}{11pt}\selectfont
        \noindent\textbf{Ecuación \theequation. #2}\\
        \noindent \hphantom{.}Fuente: #3\\
      } 
    }
% para poner abajo-centro el numero de pagina en los indices
    \usepackage{afterpage}
% quitar monoespaciado en links
    \AtBeginDocument{%
        \urlstyle{same} 
    }
% quitar subtitulos de los apendices (cuestionable)
    \newcommand{\usectionNoToc}[1]{%
        \phantomsection
        \section*{#1}
    }
    \newcommand{\usubsectionNoToc}[1]{%
        \phantomsection
        \subsection*{#1.}
    }
%para combinar celdas ig
    \usepackage{tabularx}
    \usepackage{array}
%para no romper parafos
    \widowpenalties 1 10000
    \raggedbottom
%///endauri