\uextra{Apéndice}{Manual de Usuario}

\section*{Introducción}

Este sistema es una solución basada en inteligencia artificial, diseñada para identificar sonidos ambientales y generar alertas en situaciones de emergencia que puedan indicar que una persona se encuentra en una posición de vulnerabilidad.

El objetivo principal del sistema es ofrecer un monitoreo continuo, discreto y no invasivo, capaz de detectar eventos sonoros anómalos (gritos de auxilio, caídas, rotura de cristales o silencios prolongados) y notificar de manera inmediata a contactos de emergencia predefinidos.

El sistema integra tres componentes clave:

\begin{itemize}
  \item Un módulo de clasificación acústica que utiliza modelos preentrenados como YAMNet y Vosk para identificar sonidos y palabras clave.
  \item Un módulo de detección de anomalías basado en modelos de inteligencia artificial como LSTM e Isolation Forest, que analiza secuencias de sonidos para identificar comportamientos atípicos.
  \item Un módulo de notificaciones que envía alertas en tiempo real a través de Telegram, correo electrónico o SMS.
\end{itemize}

Este manual contiene las instrucciones necesarias para interactuar con el sistema, comprender sus señales y responder adecuadamente en diferentes situaciones.

\section*{Desarrolladores del sistema}

El Sistema de Monitoreo Acústico fue desarrollado por los tesistas Carmelo Naim y César Sotillo, estudiantes de Ingeniería Informática de la Universidad Católica Andrés Bello, como parte de su Trabajo de Grado, realizado entre 2024 y octubre de 2025.

\section*{Requisitos previos}

Antes de comenzar a utilizar el sistema, asegúrese de cumplir con los siguientes requisitos:

Conocimiento previo:

\begin{itemize}
  \item No se requieren conocimientos técnicos avanzados.
  \item Es recomendable familiarizarse con los comandos de voz que el sistema reconoce.
\end{itemize}

Requisitos del sistema:

\begin{itemize}
  \item Hardware: Dispositivo Raspberry Pi 4, micrófonos USB, conexión estable a internet y energía eléctrica continua
  \item Software: Aplicación de mensajería Telegram instalada en el teléfono del usuario o contacto designado.
  \item Red: Conexión Wi-Fi o Ethernet.
\end{itemize}

\section*{Requisitos previos}

El sistema opera de manera autónoma y no requiere una interfaz gráfica para su funcionamiento cotidiano. Sin embargo, el usuario puede interactuar con él mediante comandos de voz y recibir notificaciones.

Componentes principales:

\begin{itemize}
  \item Agentes de audio: Programas que escuchan continuamente el ambiente y clasifican sonidos.
  \item Sistema de alertas: Envía notificaciones automáticas cuando detecta una posible emergencia.
  \item Interacción por voz: Permite al usuario cancelar alertas o confirmar su estado.
\end{itemize}

Flujo de interaccion:

\begin{itemize}
  \item El agente de audio escucha continuamente el ambiente.
  \item Al detectar un sonido clasifica.
  \item Si se identifica una anomalia, se activa el sistema de alertas.
  \item El usuario recibe una notificación.
  \item El usuario puede interactuar por voz para cancelar la alerta y confirmar su estado.
\end{itemize}

\section*{Problemas comunes}

A continuación, se presentan algunos problemas que pueden surgir durante el uso del sistema y sus posibles soluciones:

\begin{enumerate}
  \item El sistema no responde a comandos de voz
  \begin{itemize}
    \item Causa: Ruido ambiental alto o micrófono obstruido.
    \item Solución: Hable en un tono claro y dirigido hacia el micrófono. Reduzca el ruido de fondo si es posible.
  \end{itemize}
  \item Falsos positivos
  \begin{itemize}
    \item Causa: El sistema puede estar interpretando eventos comunes como anomalías.
    \item Solución: Cancela la alerta usando el comando de voz "Estoy bien".
  \end{itemize}
  \item No se reciben notificaciones en Telegram
  \begin{itemize}
    \item Causa: Problemas de conexión a Internet.
    \item Solución: Verifique la conexión a Internet.
  \end{itemize}
\end{enumerate}

\section*{Notas importantes}
\begin{itemize}
  \item Asegúrese de que el micrófono esté correctamente conectado y configurado en su dispositivo. Además de estar colocado en un lugar adecuado para captar el sonido.
  \item Mantenga el software del sistema actualizado el software del sistema para mejorar la precisión en la detección.
  \item El sistema no almacena grabaciones de audio. Todo el procesamiento se realiza de forma local y en tiempo real.
\end{itemize}